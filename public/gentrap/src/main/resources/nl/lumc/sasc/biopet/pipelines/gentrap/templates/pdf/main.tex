\documentclass[a4paper,12pt]{article}
\usepackage[a4paper,margin=1in]{geometry}
\usepackage[T1]{fontenc}
\usepackage[usenames,dvipsnames]{xcolor}
\usepackage{longtable}
\usepackage{graphicx}
\usepackage{subfig}
\usepackage{listings}
\usepackage{verbatim}
\usepackage{multirow}
\usepackage{url}
\usepackage{grffile}
\usepackage[superscript]{cite}
% must be located here, so we can handle filenames with underscores
\newcommand{\UnderscoreCommands}{\do\IfFileExists \do\verbatiminput%
    \do\verbatimtabinput \do\citeNP \do\citeA \do\citeANP \do\citeN%
    \do\shortcite \do\shortciteNP \do\shortciteA \do\shortciteANP%
    \do\shortciteN \do\citeyear \do\citeyearNP%
}
\usepackage[strings]{underscore}

\usepackage{fancyhdr}
\usepackage{hyperref}

\setlength{\tabcolsep}{20pt}
\renewcommand{\arraystretch}{1.3}
\renewcommand{\familydefault}{\sfdefault}

% requires the titling package, for subtitles
%\newcommand{\subtitle}[1]{
%    \posttitle{
%        \par\end{center}
%        \begin{center}\large#1\end{center}
%       \vskip0.5em}}

\pagestyle{fancy}
\setlength{\headheight}{15.2pt}

\fancyhf{}
\fancyhead[LE,RO]{\thepage}
\fancyhead[RE]{\textit{\nouppercase{\leftmark}}}
\fancyhead[LO]{\textit{\nouppercase{\rightmark}}}

\begin{document}
\setlength{\parindent}{0in}
%\title{\Huge Gentrap Run Report}
\title{\resizebox{0.7\linewidth}{!}{\itshape Gentrap Run Report}}
\author{LUMC Sequencing Analysis Support Core}
\maketitle
\begin{center}
    {\LARGE version ((( run.version )))}
\end{center}
\begin{figure}[h!]
    \centering
    \includegraphics[width=0.8\textwidth]{((( run.logo )))}
\end{figure}
\thispagestyle{empty}
\clearpage

\addtocontents{toc}{\protect\hypertarget{toc}{}}
\tableofcontents
\clearpage


\part{Overview}
\label{sec:intro}

This document outlines the results obtained from running Gentrap, a generic
pipeline for transcriptome analysis. The pipeline itself is composed of several
programs, listed in Table~\ref{tab:programs}. Note that the list only contains
the programs used in this pipeline run.

\begin{center}
    \captionof{table}{Programs in Gentrap}
    \label{tab:programs}
    \begin{longtable}{ l l l p{0.2\textwidth} }
            \hline
            Program & Version & Checksum & Usage\\
            \hline \hline
        \endhead
            \hline
            \multicolumn{3}{c}{\textit{Continued on next page}}\\
            \hline
        \endfoot
            \hline
        \endlastfoot
        Gentrap & ((( run.version ))) & - & the full pipeline\\
        ((* for program, info in run.executables.items()|sort *))
            ((( program ))) & ((( info.version ))) & ((( info.md5|truncate(7, True, "") ))) & ((( info.desc )))\\
        ((* endfor *))
    \end{longtable}
\end{center}
% HACK: to keep table counters in sync
\addtocounter{table}{-1}

\clearpage

((* for sample in run.samples.values() *))
((* include "sample.tex" *))
\clearpage
((* endfor *))


\part{About Gentrap}
\label{apx:about}

The Generic Transcriptome Analysis Pipeline (Gentrap) is a
generic pipeline for analyzing transcripts from RNA-seq experiments. \\

Gentrap was developed by Wibowo Arindrarto (\href{mailto:w.arindrarto@lumc.nl}{w.arindrarto@lumc.nl})
based on raw scripts written by Jeroen Laros
(\href{mailto:j.f.j.laros@lumc.nl}{j.f.j.laros@lumc.nl}) and
Peter-Bram 't Hoen
(\href{mailto:p.a.c._t_hoen@lumc.nl}{p.a.c._t_hoen@lumc.nl}) as part of the
\href{https://git.lumc/nl/biopet/biopet}{Biopet framework}. \\

The Biopet framework is developed by the
\href{http://sasc.lumc.nl}{Sequencing Analysis Support Core} of the
\href{http://lumc.nl}{Leiden University Medical Center}, by extending the
\href{http://http://gatkforums.broadinstitute.org/discussion/1306/overview-of-queue}{Queue framework}.
Please see the respective web sites for licensing information.

\indent

Cover page image: T7 RNA Polymerase and a dsDNA template (PDB ID \texttt{1msw}).
Created by Thomas Splettstoesser, taken from
\href{http://commons.wikimedia.org/wiki/File:T7_RNA_polymerase.jpg}{Wikimedia Commons}.


\end{document}
