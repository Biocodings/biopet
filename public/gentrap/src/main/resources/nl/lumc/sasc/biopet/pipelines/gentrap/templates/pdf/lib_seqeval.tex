\subsection{Sequencing Results Evaluation}
\label{sec:seq}

This section contains statistics of the raw sequencing results. Statistics of
the preprocessing step(s) may be shown as well, depending on which preprocessing
steps were performed.
(Table~\ref{tab:pipelineparams}).

\indent

All statistics, except for the per sequence \%GC graph, were collected using
FastQC. Visit the
\href{http://www.bioinformatics.babraham.ac.uk/projects/fastqc/}{FastQC website}
for more detailed explanations of them.

\subsubsection{Overview}
\label{subsec:seq-overview}
There are four types of preprocessing that may be done:

\begin{description}
    \item[\textit{none}] \hfill \\
        No preprocessing step.
    \item[\textit{adapter clipping}] \hfill \\
        Removal of known adapter sequences present in the sequencing
        reads. The list of sequences are retrieved from the FastQC contaminant
        list, which is packaged with the FastQC released used in this pipeline.
    \item[\textit{quality trimming}] \hfill \\
        Removal of all low-quality bases that are often found in the 5'
        or 3' ends of the reads.
    \item[\textit{adapter clipping followed by quality trimming}] \hfill \\
        Both adapter clipping and base quality trimming.
\end{description}
\indent
Your chosen preprocessing method was:\textbf{
((* if lib.clipping *))adapter clipping
((* elif lib.trimming *))quality trimming
((* elif lib.clipping and lib.trimming *))adapter clipping followed by quality trimming((* endif *))}.

((* if lib.clipping *))
\subsubsection{Adapter removal}
Known adapter sequences found in the raw data are listed in
Table~\ref{tab:adapters}. For each adapter sequence, the count of its
occurence (partially or whole) is also listed. The presence of these
adapters do not always result in the FASTQ records to be discarded.
FASTQ records are only discarded if clipping of the adapter sequence
results in sequences shorter than the threshold set in cutadapt.

\indent

For the complete list of known adapter sequences, consult the
\href{http://www.bioinformatics.babraham.ac.uk/projects/fastqc/}{official FastQC website}.
More information about clipping using cutadapt is available on the
\href{https://code.google.com/p/cutadapt/}{official cutadapt website}.

%\ClipContamTable
\begin{center}
    \captionof{table}{Adapter Sequences Present in the Sample}
    \label{tab:adapters}
    \begin{longtable}{ p{14mm} r p{0.4\textwidth} r }
            \hline
            Read & Discarded & Adapter & Occurence\\
            \hline \hline
        \endhead
            \hline
            \multicolumn{4}{c}{\textit{Continued on next page}}\\
            \hline
        \endfoot
            \hline
        \endlastfoot
        ((* if lib.flexiprep.stats.clipping_R1 *))
            ((* if lib.flexiprep.stats.clipping_R1.adapters *))
                Read 1 & ((( lib.flexiprep.stats.clipping_R1.num_reads_affected|nice_int )))
                ((* for adapter, stat in lib.flexiprep.stats.clipping_R1.adapters.iteritems() *))
                ((* if loop.first *))
                    & ((( adapter ))) & ((( stat[1]|nice_int )))\\
                ((* else *))
                    & & ((( adapter ))) & ((( stat[1]|nice_int )))\\
                ((* endif *))
                ((* endfor *))
            ((* else *))
                Read 1 & 0 & \textit{none found} & 0\\
            ((* endif *))
            ((* if lib.is_paired_end *))
                ((* if lib.flexiprep.stats.clipping_R2.adapters *))
                    Read 2 & ((( lib.flexiprep.stats.clipping_R2.num_reads_affected|nice_int )))
                    ((* for adapter, stat in lib.flexiprep.stats.clipping_R2.adapters.iteritems() *))
                    ((* if loop.first *))
                        & ((( adapter ))) & ((( stat[1]|nice_int )))\\
                    ((* else *))
                        & & ((( adapter ))) & ((( stat[1]|nice_int )))\\
                    ((* endif *))
                    ((* endfor *))
                ((* else *))
                    Read 2 & 0 & \textit{none found} & 0\\
                ((* endif *))
            ((* endif *))
        ((* endif *))
    \end{longtable}
\end{center}
\addtocounter{table}{-1}

((* if lib.is_paired_end *))
After clipping, all read pairs are then checked ('synced') for their
completeness. Read pairs whose other half has been discarded during
clipping, will also be discarded. The summary of this step is available in
Table~\ref{tab:clipsync}.

\begin{center}
    \captionof{table}{Summary of Post-Clipping Sync Step}
    \label{tab:clipsync}
    \begin{tabular}{ l  r }
        \hline
        Parameter & Count\\ \hline \hline
        Discarded FASTQ records from read 1 & ((( lib.flexiprep.stats.fastq_sync.num_reads_discarded_R1|nice_int )))\\
        Discarded FASTQ records from read 2 & ((( lib.flexiprep.stats.fastq_sync.num_reads_discarded_R2|nice_int )))\\
        \hline
        Total kept FASTQ records & ((( lib.flexiprep.stats.fastq_sync.num_reads_kept|nice_int )))\\
        \hline
    \end{tabular}
\end{center}
((* endif *))
((* endif *))

\vspace{2mm}
((* if lib.trimming and lib.is_paired_end *))
\subsubsection{Base quality trimming}
    Summary of the trimming step is available in Table~\ref{tab:trim}. In short,
    sickle is used to trim the 5' and 3' ends of each FASTQ records so that low
    quality bases are trimmed off. If after trimming the FASTQ record becomes
    shorter than the threshold set by sickle, the entire sequence is discarded. For this
    step, read pair completeness check was done along with trimming.

    \indent

    More information about sickle is available on the
    \href{https://github.com/najoshi/sickle}{official sickle website}.

    \begin{center}
        \captionof{table}{Summary of quality trimming step}
        \label{tab:trim}
        \begin{tabular}{ l  r }
            \hline
            Parameter & Count\\ \hline \hline
            Discarded FASTQ records from read 1 & ((( lib.flexiprep.stats.trimming_R1.num_reads_discarded|nice_int )))\\
            Discarded FASTQ records from read 2 & ((( lib.flexiprep.stats.trimming_R2.num_reads_discarded|nice_int )))\\
            \hline
        \end{tabular}
    \end{center}
((* endif *))
%\vspace{2mm}

\subsubsection{Basic statistics}
%\label{subsec:seq_basic}

Basic statistics on the FASTQ files are shown below. For paired-end reads, the
read count numbers of read 1 and read 2, both before and after preprocessing,
must match. Read lengths are likely to vary after preprocessing, due to selective
clipping and trimming of the reads.

\begin{center}
    \captionof{table}{Basic Run Statistics}
    %\label{tab:basestats}
((* if lib.clipping or lib.trimming *))
        \begin{longtable}{ l r r }
                \hline
                Parameter & Raw & Preprocessed \\ \hline \hline
                \hline \hline
            \endhead
                \hline
                \multicolumn{3}{c}{\textit{Continued on next page}}\\
                \hline
            \endfoot
                \hline
            \endlastfoot
            Read 1 count & ((( lib.fastqc_r1.basic_statistics.data["Total Sequences"] ))) & ((( lib.fastqc_r1_qc.basic_statistics.data["Total Sequences"] )))\\
            Read 1 overall \%GC & ((( lib.fastqc_r1.basic_statistics.data["%GC"] ))) & ((( lib.fastqc_r1_qc.basic_statistics.data["%GC"] )))\\
            Read 1 length range & ((( lib.fastqc_r1.basic_statistics.data["Sequence length"] ))) & ((( lib.fastqc_r1_qc.basic_statistics.data["Sequence length"] )))\\
            \hline
    ((* if lib.is_paired_end *))
            Read 2 count & ((( lib.fastqc_r2.basic_statistics.data["Total Sequences"] ))) & ((( lib.fastqc_r2_qc.basic_statistics.data["Total Sequences"] )))\\
            Read 2 overall \%GC & ((( lib.fastqc_r2.basic_statistics.data["%GC"] ))) & ((( lib.fastqc_r2_qc.basic_statistics.data["%GC"] )))\\
            Read 2 length range & ((( lib.fastqc_r2.basic_statistics.data["Sequence length"] ))) & ((( lib.fastqc_r2_qc.basic_statistics.data["Sequence length"] )))\\
            \hline
    ((* endif *))
((* else *))
        \begin{longtable}{ l r }
                \hline
                Parameter & Raw \\ \hline \hline
                \hline \hline
            \endhead
                \hline
                \multicolumn{2}{c}{\textit{Continued on next page}}\\
                \hline
            \endfoot
                \hline
            \endlastfoot
            Read 1 count & ((( lib.fastqc_r1.basic_statistics.data["Total Sequences"] )))\\
            Read 1 overall \%GC & ((( lib.fastqc_r1.basic_statistics.data["%GC"] )))\\
            Read 1 length range & ((( lib.fastqc_r1.basic_statistics.data["Sequence length"] )))\\
            \hline
    ((* if lib.is_paired_end *))
            Read 1 count & ((( lib.fastqc_r1.basic_statistics.data["Total Sequences"] )))\\
            Read 1 overall \%GC & ((( lib.fastqc_r1.basic_statistics.data["%GC"] )))\\
            Read 1 length range & ((( lib.fastqc_r1.basic_statistics.data["Sequence length"] )))\\
            \hline
    ((* endif *))
((* endif *))
        \end{longtable}
\end{center}

% sequence length distribution
\subsubsection{Read length distribution}
    Read length distribution for the raw read pair data are shown below.
    Depending on your chosen preprocessing step, the length distribution for the
    preprocessed data may be shown as well. The length distribution for the
    preprocessed data usually becomes less uniform compared to the raw data due
    to the variable removal of the bases in each read.
    \begin{figure}[h!]
        \centering
        \begin{minipage}[b]{0.48\textwidth}
            \centering
            \subfloat[Raw read 1]{
                \includegraphics[width=\textwidth]{((( lib.fastqc_r1_files.plot_sequence_length_distribution.path )))}
            }
        \end{minipage}
        \begin{minipage}[b]{0.48\textwidth}
            \centering
            ((* if lib.trimming or lib.clipping *))
            \subfloat[Preprocessed read 1]{
                \includegraphics[width=\textwidth]{((( lib.fastqc_r1_qc_files.plot_sequence_length_distribution.path )))}
            }
            ((* endif *))
        \end{minipage}
        \caption{Read length distribution for read 1.}
        %\label{fig:length_dist_before_and_after_1}
    \end{figure}

((* if lib.is_paired_end *))
    \begin{figure}[h!]
        \centering
        \begin{minipage}[b]{0.48\textwidth}
            \centering
            \subfloat[Raw read 2]{
            \includegraphics[width=\textwidth]{((( lib.fastqc_r2_files.plot_sequence_length_distribution.path )))}
            }
        \end{minipage}
        \begin{minipage}[b]{0.48\textwidth}
            \centering
            ((* if lib.trimming or lib.clipping *))
            \subfloat[Preprocessed read 2]{
                \includegraphics[width=\textwidth]{((( lib.fastqc_r2_qc_files.plot_sequence_length_distribution.path )))}
            }
            ((* endif *))
        \end{minipage}
        \caption{Read length distribution for read 2.}
        %\label{fig:length_dist_before_and_after_2}
    \end{figure}
((* endif *))

% per base sequence quality
\subsubsection{Per base sequence quality}
    Here, the sequence quality for different base positions in all read pairs
    are shown. For each figure, the central line represents the median value,
    the blue represents the mean, the yellow box represents the inter-quartile
    range (25\%-75\%), and the upper and lower whiskers represent the 10\% and
    90\% points respectively. The green-shaded region marks good quality calls,
    orange-shaded regins mark reasonable quality calls, and red-shaded regions
    mark poor quality calls.

    \indent

    Note that the latter base positions shown in the figures are
    sometimes aggregates of multiple positions.
    \begin{figure}[h!]
        \centering
        \begin{minipage}[b]{0.48\textwidth}
            \centering
            \subfloat[Raw read 1]{
                \includegraphics[width=\textwidth]{((( lib.fastqc_r1_files.plot_per_base_quality.path )))}
            }
        \end{minipage}
        \begin{minipage}[b]{0.48\textwidth}
            \centering
            ((* if lib.trimming or lib.clipping *))
            \subfloat[Preprocessed read 1]{
                \includegraphics[width=\textwidth]{((( lib.fastqc_r1_qc_files.plot_per_base_quality.path )))}
            }
            ((* endif *))
        \end{minipage}
        \caption{Per base quality before and after processing read 1.}
        %\label{fig:per_base_qual_before_and_after_1}
    \end{figure}

((* if lib.is_paired_end *))
    \begin{figure}[h!]
        \centering
        \begin{minipage}[b]{0.48\textwidth}
            \centering
            \subfloat[Raw read 2]{
            \includegraphics[width=\textwidth]{((( lib.fastqc_r2_files.plot_per_base_quality.path )))}
            }
        \end{minipage}
        \begin{minipage}[b]{0.48\textwidth}
            \centering
            ((* if lib.trimming or lib.clipping *))
            \subfloat[Preprocessed read 2]{
                \includegraphics[width=\textwidth]{((( lib.fastqc_r2_qc_files.plot_per_base_quality.path )))}
            }
            ((* endif *))
        \end{minipage}
        \caption{Per base quality before and after processing for read 2.}
        %\label{fig:per_base_qual_before_and_after_2}
    \end{figure}
((* endif *))

% per sequence quality scores
\subsubsection{Per sequence quality scores}
    The read quality score distributions are shown below.
    \begin{figure}[h!]
        \centering
        \begin{minipage}[b]{0.48\textwidth}
            \centering
            \subfloat[Raw read 1]{
                \includegraphics[width=\textwidth]{((( lib.fastqc_r1_files.plot_per_sequence_quality.path )))}
            }
        \end{minipage}
        \begin{minipage}[b]{0.48\textwidth}
            \centering
            ((* if lib.trimming or lib.clipping *))
            \subfloat[Preprocessed read 1]{
                \includegraphics[width=\textwidth]{((( lib.fastqc_r1_qc_files.plot_per_sequence_quality.path )))}
            }
            ((* endif *))
        \end{minipage}
        \caption{Per sequence quality scores before and after preprocessing read 1.}
        %\label{fig:per_seq_qual_before_and_after_1}
    \end{figure}

((* if lib.is_paired_end *))
    \begin{figure}[h!]
        \centering
        \begin{minipage}[b]{0.48\textwidth}
            \centering
            \subfloat[Raw read 2]{
                \includegraphics[width=\textwidth]{((( lib.fastqc_r2_files.plot_per_sequence_quality.path )))}
            }
        \end{minipage}
        \begin{minipage}[b]{0.48\textwidth}
            \centering
            ((* if lib.trimming or lib.clipping *))
            \subfloat[Preprocessed read 2]{
                \includegraphics[width=\textwidth]{((( lib.fastqc_r2_qc_files.plot_per_sequence_quality.path )))}
            }
            ((* endif *))
        \end{minipage}
        \caption{Per sequence quality scores before and after preprocessing for read 2.}
        %\label{fig:per_seq_qual_before_and_after_2}
    \end{figure}
((* endif *))

% per base sequence content
\subsubsection{Per base sequence content}
    The figures below plot the occurence of each nucleotide in each position in
    the reads. In a completely random library, you would expect the differences
    among each nucleotide to be minor. You may sometimes see a skewed
    proportion of nucleotides near the start of the read due to the use of
    non-random cDNA during sample preparation.

    \indent

    Note that the latter base positions shown in the figures are
    sometimes aggregates of multiple positions.
    \begin{figure}[h!]
        \centering
        \begin{minipage}[b]{0.48\textwidth}
            \centering
            \subfloat[Raw read 1]{
                \includegraphics[width=\textwidth]{((( lib.fastqc_r1_files.plot_per_base_sequence_content.path )))}
            }
        \end{minipage}
        \begin{minipage}[b]{0.48\textwidth}
            \centering
            ((* if lib.trimming or lib.clipping *))
            \subfloat[Preprocessed read 1]{
                \includegraphics[width=\textwidth]{((( lib.fastqc_r1_qc_files.plot_per_base_sequence_content.path )))}
            }
            ((* endif *))
        \end{minipage}
        \caption{Per base sequence content before and after preprocessing for read 1.}
        %\label{fig:per_base_content_before_and_after_1}
    \end{figure}

((* if lib.is_paired_end *))
    \begin{figure}[h!]
        \centering
        \begin{minipage}[b]{0.48\textwidth}
            \centering
            \subfloat[Raw read 2]{
                \includegraphics[width=\textwidth]{((( lib.fastqc_r2_files.plot_per_base_sequence_content.path )))}
            }
        \end{minipage}
        \begin{minipage}[b]{0.48\textwidth}
            \centering
            ((* if lib.trimming or lib.clipping *))
            \subfloat[Preprocessed read 2]{
                \includegraphics[width=\textwidth]{((( lib.fastqc_r2_qc_files.plot_per_base_sequence_content.path )))}
            }
            ((* endif *))
        \end{minipage}
        \caption{Per base sequence content before and after preprocessing for read 2.}
        %\label{fig:per_base_content_before_and_after_2}
    \end{figure}
((* endif *))


% per sequence GC content
\subsubsection{Per sequence GC content}
    The figures below show the GC percentage distribution of all the read pair
    data.
    \begin{figure}[h!]
        \centering
        ((* if lib.trimming or lib.clipping *))
        \begin{minipage}[b]{0.48\textwidth}
            \centering
            \subfloat[Raw read 1]{
                \includegraphics[width=\textwidth]{((( lib.fastqc_r1_files.plot_per_sequence_gc_content.path )))}
            }
        \end{minipage}
        \begin{minipage}[b]{0.48\textwidth}
            \centering
            \subfloat[Preprocessed read 1]{
                \includegraphics[width=\textwidth]{((( lib.fastqc_r1_qc_files.plot_per_sequence_gc_content.path )))}
            }
        \end{minipage}
        ((* else *)))
        \begin{minipage}[b]{0.48\textwidth}
            \centering
            \subfloat[Raw read 1]{
                \includegraphics[width=\textwidth]{((( lib.fastqc_r1_files.plot_per_sequence_gc_content.path )))}
            }
        \end{minipage}
        ((* endif *))
        \caption{Per sequence GC content before and after preprocessing for read 1.}
        %\label{fig:per_base_content_before_and_after_1}
    \end{figure}

((* if lib.is_paired_end *))
    \begin{figure}[h!]
        \centering
        ((* if lib.trimming or lib.clipping *))
        \begin{minipage}[b]{0.48\textwidth}
            \centering
            \subfloat[Raw read 2]{
                \includegraphics[width=\textwidth]{((( lib.fastqc_r2_files.plot_per_sequence_gc_content.path )))}
            }
        \end{minipage}
        \begin{minipage}[b]{0.48\textwidth}
            \centering
            \subfloat[Preprocessed read 2]{
                \includegraphics[width=\textwidth]{((( lib.fastqc_r2_qc_files.plot_per_sequence_gc_content.path )))}
            }
        \end{minipage}
        ((* else *))
        \begin{minipage}[b]{0.48\textwidth}
            \centering
            \subfloat[Raw read 2]{
                \includegraphics[width=\textwidth]{((( lib.fastqc_r2_files.plot_per_sequence_gc_content.path )))}
            }
        \end{minipage}
        ((* endif *))
        \caption{Per sequence GC content before and after preprocessing for read 2.}
        %\label{fig:per_base_content_before_and_after_2}
    \end{figure}
((* endif *))


